\documentclass[a4paper, 12pt]{scrartcl}

\usepackage[T1]{fontenc}
%\usepackage[utf8]{inputenc}
\usepackage{fontspec}
\usepackage[ngerman]{babel}
\usepackage{setspace}
\usepackage{fancyhdr}
\usepackage{titling}
\usepackage{titlesec}
\usepackage{graphicx}
\usepackage{wrapfig}
\usepackage[stable]{footmisc}

%\usepackage[modulo]{lineno}
\usepackage{lineno}

\usepackage{tikz}
\usetikzlibrary{positioning}

\pagestyle{fancy}
\fancyhf{}
\fancyhead[L]{}
\fancyhead[C]{\nouppercase{\rightmark} (\nouppercase{\leftmark})}
\fancyhead[R]{}
\fancyfoot[C]{\thepage}
\renewcommand{\chaptermark}[1]

\renewcommand{\headrulewidth}{0.4pt}
\renewcommand{\footrulewidth}{0.4pt}

\renewcommand{\familydefault}{\rmdefault}


\addtokomafont{labelinglabel}{\sffamily}

%\renewcommand{\maketitle}{
%  \begin{center}
%    {\huge \bfseries \thetitle}\\
%    \vspace{.25em}
%    {\Large \theauthor}
%  \end{center}
%}

\titleformat{\section}
{\Large \bfseries}
{}
{0em}
{}[\titlerule]

\titleformat{\subsection}
{\large \bfseries}
{}
{0em}
{}

\titleformat{\subsubsection}[runin]
{\bfseries}
{}
{0em}
{}

\usepackage{listings}
\usepackage{color}

\definecolor{dkgreen}{rgb}{0,0.6,0}
\definecolor{gray}{rgb}{0.5,0.5,0.5}
\definecolor{mauve}{rgb}{0.58,0,0.82}

\lstset{frame=tb,
  language=Java,
  numbers=left,
  aboveskip=3mm,
  belowskip=3mm,
  showstringspaces=false,
  columns=flexible,
  basicstyle={\small\ttfamily},
  numberstyle=\tiny\color{gray},
  keywordstyle=\color{blue},
  commentstyle=\color{dkgreen},
  stringstyle=\color{mauve},
  breaklines=true,
  breakatwhitespace=true,
  tabsize=3
}

\title{39. Bundeswettbewerb Informatik, 1. Runde}
\author{Jan Ehehalt, Jonathan Hager}
\date{}

\begin{document}
\maketitle
\newpage
\tableofcontents
\newpage

\section{Aufgabe 1 - Wörter aufräumen}

Im Folgenden sieht man ein Codebeispiel für Code, der beispielhaft geschrieben wurde.

\begin{lstlisting}
// Kommentare sind ne geile Sache
import com.badlogic.ShitRenderer;

public static void main(String[] args){
    Controller.control(theWorld);
    Math.atan(1/0);
}
\end{lstlisting}

Im Codebeispiel sieht man, wie der JavaScript Compiler intern arbeitet. Besondere Achtung sollte hierbei dem Math-Befehl gegeben werden, denn Math wurde nicht importiert und deshalb crasht es bereits deshalb, nicht wegen der ZeroDivision, da JavaScript hier einfach das Ergebnis würfeln wurde. \# Ehre

\section{Aufgabe 2 - Dreieckspuzzle}
\subsection{Idee}
\subsection{Erläuterung des Codes}

\section{Aufgabe 3 - Tobis Turnier}
\section{Aufgabe 4 - Streichholzrätsel}
\section{Aufgabe 5 - Wichteln}

%\begin{linenumbers}\resetlinenumber
%\end{linenumbers}

\end{document}
